
%%%% Eingebundene Pakete %%%% 
\usepackage{xltxtra}

\usepackage{scrhack} 
\usepackage{xcolor}
\usepackage{color}
\usepackage{xfrac} % \sfrac{}{} für Brueche
\usepackage{graphicx}
\usepackage{float}
\usepackage{subcaption}
\usepackage{setspace}
\usepackage{wrapfig}
\usepackage{longtable}
\usepackage{geometry}
\usepackage{scrlayer-scrpage}
\usepackage{wallpaper}
\usepackage{ulem}
\usepackage{siunitx}
\usepackage{amsmath}
\usepackage{chngcntr}

%% Sorgt dafür, dass Kapitel neue Kapitel nicht auf einer neuen rechten Seite anfangen
\RedeclareSectionCommand[style=section,indent=0pt]{chapter}

%% Definition der Auto Intern Farben (Das AIschwarz wir im Fließtext nicht benutzt, stattdessen ist normales schwarz gesetzt (schwarze Tinte beim Druck günstiger)
\definecolor{THrot}{RGB}{228,0,30}
\definecolor{THblau}{RGB}{0,61,125}



%% Auto-Intern vereinfachte Befehle für Schriftarten
\newcommand{\mainfont}{\setmainfont{Lato-Regular.ttf}}
\newcommand{\LatoReg}{\setmainfont{Lato-Regular.ttf}}
\newcommand{\DaysOne}{\setmainfont{DaysOne-Regular.ttf}}
\newcommand{\LatoBold}{\setmainfont{Lato-Bold.ttf}}
\newcommand{\Avenir}{\setmainfont{AvenirLTStd-Book.otf}}
\newcommand{\LatoBlack}{\setmainfont{Lato-Black.ttf}}

%% Auto-Intern vereinfachte Befhele für Sprachwechsel eng - de
\newcommand{\Deutsch}{\usepackage[ngerman]{babel}
\renewcaptionname{ngerman}{\contentsname}{Inhaltsverzeichnis} % Standard: Inhaltsverzeichnis
\renewcaptionname{ngerman}{\listfigurename}{Abbildungsverzeichnis} % Standard: Abbildungsverzeichnis
\renewcaptionname{ngerman}{\listtablename}{Tabellenverzeichnis} % Standard: Tabellenverzeichnis
\renewcaptionname{ngerman}{\figurename}{Abbildung} % Standard: Abbildung
\renewcaptionname{ngerman}{\tablename}{Tabelle} % Standard: Tabelle
}


%% Auto-Intern vereinfachte Befehle für Textsatz
\newcommand{\Kursiv}[1]{\setmainfont{Lato-Italic.ttf}#1 \mainfont}
\newcommand{\Fett}[1]{\setmainfont{Lato-Bold.ttf}#1 \mainfont}
\newcommand{\FettKursiv}[1]{\setmainfont{Lato-BoldItalic.ttf}#1 \mainfont}
\newcommand{\Hoch}[1]{$^{\text{#1}}$}
\newcommand{\Tief}[1]{$_{\text{#1}}$}

%% Auto-Intern vereinfacht Befehle für Kapitel usw, um differenzierung zwischen Inhaltsverzeichnis und Text zu haben
\newcommand{\AItitlefont}{\LatoBold}
\newcommand{\Chapter}[1]{\chapter[#1]{\AItitlefont #1}}
\newcommand{\Section}[1]{\section[#1]{\AItitlefont #1}}
\newcommand{\Subsection}[1]{\subsection[#1]{\AItitlefont #1}}
\newcommand{\Subsubsection}[1]{\subsubsection[#1]{\AItitlefont #1}}
\newcommand{\Paragraph}[1]{\paragraph[#1]{\AItitlefont #1}}
\newcommand{\Subparagraph}[1]{\subparagraph[#1]{\AItitlefont #1}}

%% Einstellungen für Kopf- und Fußzeilen
\newcommand{\HeadAVier}{
\setlength{\footheight}{-1cm}
\setlength{\skip\footins}{5mm}
\setkomafont{pageheadfoot}{\setmainfont{Lato-Regular.ttf}} %Schriftart für Kopfzeile
\setkomafont{pagehead}{\setmainfont{Lato-Regular.ttf}} % Schriftart für Fußzeile
\pagestyle{scrheadings} % Seitenstil
\ihead{\includegraphics[height=2cm]{../TemplateGraphics/Logo300.jpg}} % Innere Kopfzeile 7cm
\ohead{}
\chead{}
}


\newcommand{\ImpFootAVier}{
\ofoot{\LatoBlack \fach} % Äußere Fußzeile
\cfoot{}
\ifoot{
}}


\newcommand{\NormFoot}{
	\ifoot{\scriptsize% Innere Fußzeile
	\begin{tabular}{ll}
	\matrikel - \name\\%
	\matrikelpartner - \namepartner\\%
	Betreut durch: \betreuer
	\end{tabular}
	}
	\ofoot{\LatoBlack \fach\\ \versuchsbezeichnung \semester}
	\cfoot{\pagemark}
}

\newcommand{\EmptyFoot}{
\ifoot{}
\ofoot{}
\cfoot{}
}
%% Anderung der Schriftarten für Titelbeschriftungen
\usepackage{tocloft}
\setkomafont{disposition}{\AItitlefont\color{THblau}}  % farbe: ueberschrift
\renewcommand{\cfttoctitlefont}{\LatoBold \Large \color{THblau}}
\renewcommand{\cftpartfont}{\LatoReg}
\renewcommand{\cftchapfont}{\LatoReg}
\renewcommand{\cftsecfont}{\LatoReg}
\renewcommand{\cftsubsecfont}{\LatoReg}
\renewcommand{\cftsubsubsecfont}{\LatoReg}
\renewcommand{\cftparafont}{\LatoReg}
\renewcommand{\cftsubparafont}{\LatoReg}
\renewcommand{\cftfigfont}{\LatoReg}
%\renewcommand{\cftsubfigfont}{\LatoReg}
\renewcommand{\cfttabfont}{\LatoReg}
%\renewcommand{\cftsubtabfont}{\LatoReg}



\addtocontents{toc}{\protect\thispagestyle{scrheadings}}
\newcommand{\AVier}{
\HeadAVier
\ImpFootAVier
\AItitlefont
\addtolength{\wpYoffset}{-3cm}
\ThisCenterWallPaper{0.7}{../TemplateGraphics/Kopf.pdf}%
{\color{white}.}\\
\vspace{3.5cm}
\begin{center}
{{\huge \titelname  zu \versuchsbezeichnung} \\ \LatoReg \versuchsname\\ \vspace{40mm} durchgeführt von \\ \Fett \matrikel \LatoReg \name \\ \Fett \matrikelpartner \LatoReg \namepartner \\ im \semester am \datum \\ \vspace{10mm} Betreut durch: \betreuer \\ Dozent: \dozent}
\end{center}
\newpage
\mainfont
\NormFoot
\tableofcontents
\newpage
\setcounter{page}{1}
\ohead{\thepage}
}
