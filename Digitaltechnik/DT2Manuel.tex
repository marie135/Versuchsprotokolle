

	Schaltungstechnische Realisierung eines seriellen, 4-Bit Rechenwerkes	
	
	Zu beginn des Versuchs DT2 wird, je nach Gruppe, festgelegt ob ein 4-Bit 
	Addierer oder Subtrahierer zu bearbeiten ist.
	Die unterschiedliche Aufgabenstellung, je nach Gruppe, kommt dabei erst
	in der Vorbereitenden Aufgabe 1.8 (Blockschaltbild 3) zu tragen.
	Im vorliegendem Fall wird der Subtrahierer erstellt.
	
	Die Vorbereitenden Aufgaben beginnen mit grundsätzlichem wie z.B. die 					Einarbeitung in Rechenschaltung, Flipflop-Arten(1.1) oder die Unterschiede 
	zwischen Halb- und Voll-Addierer bzw. Subtrahierer(1.3). 
	Im weiteren Verlauf muss man, nach den Regeln der Schaltungssynthese, 
	Voll- und Halb-Addierer sowie Subtrahierer entwerfen(1.2).
	Anschließend sind noch die Fragen nach der anzahl der benötigten
	D-Flipflops je Schieberegister(1.4) und der Anzahl von Tacktzyklen des 
	Systemtakts(1.5) zu beantworten.
	
	Weiterführend ist das erste Blockschaltbild zu vervollständigen.
	Hierbei wird die Verschaltung der Schieberegister zur Aufnahme der 
	Summanden, im Fall des Addierers, bzw. des Subtrahenden und Minuenden, 
	im Fall des Subtrahierers, und des Ergebnisses.
	Dabei ist speziell auf die Schalter Verkabelung zum freischalten und zum 
	löschen der Register zu achten.
	
	Im Blockschaltbild 2 wird eine Schaltung zur Generierung des Systemtaktes
	erstellt. Auch dabei ist auf den korrekten Anschluss der Schalter zu 
	achten um die vorgegebenen Funktionen zu gewährleisten. 
	
	Das dritte Blockschaltbild, Aufgabe 1.8b, zeigt die schaltung 
	des Vollsubtrahierers.
	Bedingt durch den Aufbau des Vollsubtrahierers sind einige Besonderheiten
	zu beachten. Zum einen muss die Speicherung der Entleihung gesondert im 
	Schaltbild verkabelt werden. Des weiteren soll laut Aufgabenstellung eine 
	LED eingebunden werden die im Falle eines negativen Rechenergebnisses 
	mit einer Frequenz von 2 Hz blinkt. Wie zuvor ist im allgemeinen auch wieder
	auf das anschließen der Schalter wie vorgegeben zu achten.
	
	Am Versuchstag sind die einzelnen Schaltungen gemäß der zuvor angefertigten 
	Blockschaltbilder aufzubauen und zu Prüfen.
	Dabei ist jede Teilschaltung sowie das gesamte Rechenwerk vorzuführen.
	Bei Korrekten Schaltzeichnungen sollte der praktische Aufbau der Schaltungen
	kein Problem darstellen. 