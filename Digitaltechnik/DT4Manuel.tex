Versuchsdiskussion Digitaltechnik Praktikum SS'17

Versuch DT4

In dem Versuch DT4 geht es darum, mittels eines CLPD-Bausteins, eine Sieben-Segment-Anzeige zu programmieren.
Die für jede Gruppe spezifische Code-Tabelle befindet sich in den Versuchsunterlagen. 

Zuerst sind aus der Code-Tabelle eine Wahrheitstabelle für die einzelnen Segmente A-G zu erstellen.
Anschließend sind einzelne KV-Diagramme und die dazugehörigen Schaltungsgleichungen zu erarbeiten.

Am Versuchstag selbst werden, anhand der vorbereiteten KV-Diagramme, in einem speziell dafür vorgesehenem Programm die Schaltzeichnungen der einzelnen Segmente erstellt. Im weiteren Verlauf muss man die, in den Schaltzeichnungen festgelegten, Ein- sowie Ausgänge zu den dazugehörigen Pins des CLPD-Bausteins zuweisen.

Sofern alle Arbeitsschritte wie z.B. Speichern und Kompilieren korrekt ausgeführt wurden und die Schaltzeichnungen korrekt sind funktioniert die Sieben-Segment-Anzeige wie in der Code-Tabelle vorgesehen.

Mögliche Fehlerquellen sind wie im vorliegendem Fall geschehen beispielsweise nicht korrekt verbundene Leitungen in den Zeichnungen oder auch vom Programm nicht ordnungsgemäß übernommene Pinbelegung.
