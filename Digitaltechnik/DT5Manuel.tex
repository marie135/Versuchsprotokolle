	
	DT 5 3-Bit Komparator
	
	Die Vorbereitenden Aufgaben für den Versuch DT 5 beginnen mit dem anfertigen der 
	Wahrheitstabellen für einen 1-Bit Komparator. Jeweils für einen regulären und einen 
	1-Bit Komparator mit einem zusätzlichen Sperreingang.
	Der 1-Bit Komparator mit Sperreingang ist notwendig um den 3-Bit Komparator mit 
	verkürzter Wahrheitstabelle zu erstellen. 
	Die Wahrheitstabelle des 3-Bit Komparators wird in verkürzter Form erstellt um den 
	Aufwand der einzelnen Zustände sowie der Schaltung zu verringern.
	Es ist möglich eine verkürzte Form der Wahrheitstabelle zu erstellen indem zuerst 
	das most-significant-bit(MSB), also das höchstwertige Bit, der beiden zahlen verglichen 
	wird. Erst wenn das MSB der beiden zu vergleichenden zahlen identisch ist wird das um 
	eine stelle niedrigere Bit verglichen. Dieser Prozess ist beliebig lang wiederholbar 
	in der Schaltung würde nur für jedes weitere Bit ein weiterer 1-Bit Komparator mit 
	Sperreingang angehängt werden.
	Nachdem die Schaltzeichnungen im Zuge der Vorbereitenden Aufgaben angefertigt wurden 
	sind am Versuchstag nur noch die Schaltungen auf eines der HPS-Boards zu übertragen.
	
	Der Versuch DT 5 sollte keine großen Schwierigkeiten bereiten und schnell 
	abgeschlossen sein.
	