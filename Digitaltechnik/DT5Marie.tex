In Versuch DT5 sollte ein 3-Bit Komparator entworfen werden, der am Versuchstag aufzubauen und in seiner Funkton vorzuführen war. \par
Dazu sollten in den vorbereitenden Aufgaben Wahrheitstabellen zum 1-Bit-Komparator ohne und mit zusätzlichem Eingang E entwickelt werden. Anschließend war die verkürzte Wahrheitstabelle zum 3-Bit-Komparator zu entwickeln. Es ist nur die verkürzte Wahrheitstabelle verlangt, da die Tabelle ansonsten unübersichtlich lang wäre und die anderen Fälle nicht relevant sind. An Hand der Wahrheitstabellen war schließlich eine Schaltung für einen 3-Bit-Komparator zu entwerfen. \par
Die Komparatoren haben zwei oder drei Eingänge und drei Ausgänge. Zwei Eingänge dienen der Eingabe der beiden Zahlen und der optionale dritte Eingang E ist ein Sperreingang. Nur wenn eine 1 an dem Eingang anliegt, schaltet er den Komparator frei. Der X-Ausgang zeigt an, dass die erste Zahl größer ist als die zweite, der Z-Ausgang, dass die erste Zahl kleiner ist als die zweite und der Y-Ausgang, dass beide Zahlen gleich groß sind. \par
Um einen 3-Bit-Komparator zu realisieren, benötigt man einen 1-Bit-Komparator mit zwei Eingängen für die werthöchsten Zahlen. Dann folgen zwei 1-Bit-Komparatoren mit zusätzlichem Eingang E, der nur freischaltet, wenn die werthöheren Zahlen gleich sind. Dazu wird der Y Ausgang des vorherigen Komparators an den E Eingang des folgenden angeschlossen. \par
Am Versuchstag sollte zuerst ein 1-Bit-Komparator mit zusätzlichem Eingang E auf ein HPS-Board aufgebaut und in seiner Funktion vorgeführt werden. Dieser funktionierte auf Anhieb einwandfrei. Anschließend sollte der zuvor entwickelte 3-Bit-Komparator aufgebaut werden und die Funktion vorgeführt werden. Auch dieser funktionierte ohne Probleme. Beim Aufbauen gab es keine Schwierigkeiten, da unterschiedliche Farben verwendet wurden, wodurch der komplexe Aufbau in einfachere Teilschaltungen unterteilt wurde. Fragen zu den Komparatoren und den vorbereitenden Aufgaben konnten ebenfalls beantwortet werden. \par
Die vorbereitenden Aufgaben konnten ohne größere Probleme in der vorgegebenen Zeit gelöst werden.