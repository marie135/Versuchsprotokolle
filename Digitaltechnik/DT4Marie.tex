In Versuch DT4 sollte ein BCD -> 7-Segment-Kodeumsetzer entworfen und mit Hilfe eines CPLD Bausteines realisiert werden.\par 
Jede Gruppe hat einen speziellen Kode, der umzusetzen war. Dazu sollten in den vorbereitenden Aufgaben eine Wahrheitstabelle des Kodeumsetzers und KV-Diagramme für die einzelnen Segmente erstellt werden. Aus den KV-Diagrammen wurden die speziellen Gleichungen für jedes Segment abgelesen. \par
Am Versuchstag sollte mit einem Computerprogramm die Ansteuerlogik mittels UND, ODER und NICHT Gattern entworfen werden. Es wurde für jedes Segment eine eigene Schaltung entworfen. Anschließend musste die Pin-Belegung des CPLD Bausteins festgelegt werden. Obwohl wir die vorbereitenden Aufgaben korrekt gelöst hatten, traten zwei Fehler auf. Diese sind entstanden, weil bei zwei Segmenten Verbindungen fehlten. Diese Fehler waren jedoch schnell zu beheben und nach erneutem Kompilieren funktionierte alles einwandfrei. \par
Die vorbereitenden Aufgaben konnten ohne Probleme in der vorgegebenen Zeit gelöst werden. Während dem Versuch war die Zeit jedoch sehr knapp, da man sich zunächst in das Programm einarbeiten musste. Außerdem führte ein kleiner Schreibfehler in den Unterlagen zu Verwirrung, da ein CLPD Baustein im Internet nicht zu finden war, sondern nur ein CPLD Baustein. Deshalb waren wir uns nicht sicher, ob es der richtige Baustein ist, was jedoch zu Beginn des Praktikums geklärt werden konnte.
 