Die zugrunde liegenden vorbereitenden Aufgaben beginnen damit, dass zu erst die Wahrheitstabellen
der einzelnen Flip-Flop typen erarbeitet werden. \par
Im zweiten Schritt sind die KV-Diagramme und daraus resultierend die konjunktive sowie die disjunktive 
Normalform der Schaltgleichung der FF´s zu erstellen.
Die disjunktive Normalform wird gebildet indem man die "1" Einträge aus den KV-Diagrammen zusammenfast.
Die konjunktive Normalform bildet man indem man die "0" Einträge der KV-Diagramme zusammenfast, 
alternativ ist es möglich die disjunktive Normalform, unter beauchtung der Schaltalgebraischen regeln, zu negieren.\par
Abschließend sind, aus den KV-Diagrammen und aus der konjunktiven sowie disjunktiven Normalform, schaltungen mit NOR- 
und NAND-Bausteinen zu den einzelnen Flip-Flops zu erstellen.\par
Bei dem Laborversuch selbst werden, unter zuhilfenahme eines HPS-Boards sowie IC´s und Laborleitungen, die Schaltungen nachgebaut und einer Abnahme in form eines Funktionstest unterzogen.\par
Alle gestellten Aufgaben wurden in der gegebenen Zeit erfolgreich bearbeitet.
Das einzige Problem das auftrat ergab sich durch eine Fehlinterpretation der Aufgabenstellung zu der theoretischen 
erstellung der Schaltungen. 
Es wurden nicht wie vorgesehen Schaltungen mit einem Ausgang "Q" erstellt sondern wie 
Technisch und in der gägingen Literatur üblich mit zwei Ausgängen "Q" und "nicht-Q".